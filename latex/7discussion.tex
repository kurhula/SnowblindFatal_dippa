\chapter{Conclusions}
\label{chapter:discussion}

There are so many various trades whose operation or parts of it can be abstracted to a VRP that there is much room for improvement in terms of automation in our society. It is no wonder that so much research has been done on the subject. Any sufficiently large company will probably benefit from applying it to various processes unless their everyday problems have absolutely nothing to do with VRPs.

The basic requirement for benefitting from this kind of work planning is having some sort of formalised database to which customer information is being stored. Then it is possible to transform that data and use it to solve the VRP associated with the business at hand. Naturally there are many trades which do not have anything in common with VRPs, but many aspects around the basic necessities of modern human life heavily involve concepts related to VRPs. Delivery of food and other goods from producers to intermediate storages and furthermore to markets are a prime example, and if grocery deliveries straight to the doorstep becomes more common, there will be yet another common application for VRPs.  

Even if in this specific case there was not much of a benefit in using an advanced solver, there was no downside to it either. The main benefit in this case was simply the automation process and providing a framework upon which future development can be based. A job which once required human labour has now been partly replaced by a machine. Furthermore, if the client company takes this work planning automation further, it is entirely possible to reach a point where only on specific exceptional circumstances does the process require any human interaction at all. 

Based on my experiences with this project, I am certain that automating this kind of work planning would bring great benefit to a lot of companies. Though businesses are different and each has its own unique set of requirements, versatile libraries such as Jsprit could probably be used for their purposes. The biggest work would be adapting the business model into an algorithmic format. A general solution would work only in select cases where the business model of the companies is automatically translated into a certain variation of a VRP. Otherwise customisation is needed, meaning that for small scale operations, automatisation might not be cost effective. 

The open source library Jsprit continues the good trend of increasing number of free tools being available to the general public. It is a good example how free software can be used to make human societies more efficient. This will eventually and ultimately lead to humankind having more free time as workforce is replaced by machines. Assuming that this redistribution of work is taken into account on a political level (a huge assumption that does not hold true so far, admittedly), humankind can only benefit from this development.


